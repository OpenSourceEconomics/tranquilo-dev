%!TEX root = ../mens-sana.tex
\blindmathfalse
\noindent%

    Almost by definition, maintaining high levels of physical and cognitive functioning into old age are key factors for successful ageing. Numerous studies have studied the factors influencing either physical health or cognition. However, the bio-medical literature shows both processes to be closely intertwined. On the one hand, high levels of cognitive functioning allows planning of health-related activities, gauging the consequences of actions, and adhering to medication plans. On the other hand, shocks to physical health have shown to predate declines in cognitive health \citep{Schiele.Schmitz.2021}
    % https://www.rwi-essen.de/media/content/pages/publikationen/ruhr-economic-papers/rep_21_919.pdf
    % Can use that to pick some references for above placeholders.

    In this paper, we take a broad and systematic approach to model the interdependency of physical health and cognitive functioning over the last third of the life-cycle. In order to do so, we adapt the model by \citet{Cunha.2010}, originally developed for the development of human capital during childhood and adolescence (also see for applications and extensions \citet{NixEtAl.2020, AgostinelliWiswall.2016.2, Freyberger.2021}). In our case, the approach combines factor models for physical health, cognition, and investments into both of these with a nonlinear framework to describe their evolution over time. We use the HRS data from 2002--2016, which gives us individual trajectories of physical and cognitive capacity, and investments over ages 50 to 95 years.

    Our key results indicate
    1\string) Importance of physical exercise and cognitive stimulation for maintaining physical health and cognitive functioning at later stages of life
    % (Figures \ref{fig:transition-equations-median-female} and \ref{fig:transition-equations-median-male}),
    2\string) Sizeable measurement errors across all factors
    % (Table \ref{tab:loadings_meas_sds_excerpt})
    , and 3\string) Well identified standard deviations of shocks in the model.
    %  (Table \ref{tab:shock_sds}).







    % We use the HRS data from 2002--2016, which gives us individual trajectories of physical health, cognition, and investments over ages 50 to 95 years. The variables measuring each of the three factors are described below. An individual dies if physical health falls below a threshold, which we normalise at zero.

    % Figure~\ref{fig:female_cross_factors} and Figure~\ref{fig:male_cross_factors} show that the measures form three clusters, which are highly correlated internally.

    % Table~\ref{tab:transition_female} and Table~\ref{tab:transition_male} show that the factors decline in age, especially so in the later periods.

    % Table~\ref{tab:loadings_meas_sds_female} and Table~\ref{tab:loadings_meas_sds_male} show that measurement error is sizeable for all latent factors.

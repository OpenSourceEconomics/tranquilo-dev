\subsubsection{Subsolver}
\label{subsubsec:subsolver}
\hfill\\

\noindent Recall that in each iteration $t$, we approximately solve for the minimizer of the aggregate model
$M_{t}^{s} (x_t + s)$ in $B(x_t, \Delta_t^{region})$ to obtain a new candidate step $s_t$ (see \ref{eq:cand-step}). Before solving the subproblem, we center the trust region at the current iterate $x_t$, such that the trust region center is $0$ and the radius $\Delta^{region}_{t}$ is normalized to $1$.
We distinguish between two cases: (i) the unconstrained case, where the trust region is the entire $B(0, 1)$, and (ii) the constrained case, where the trust region is a bounded subspace of $B(0, 1)$.
If (i) there are no bounds on the parameter space, the subsolver GQTPAR is used; if (ii) there
are bounds on the parameter space, we employ the Bounded Newton Trust Region (BNTR) algorithm.
In both cases, the subsolver returns the candidate step $s_t$ that minimizes the quadratic model $M_{t}^{s} (x_t + s)$
within the current trust region:


% interval $\left[ -1, 1 \right]^n$ $\text{s.t.} \ \lVert s \rVert \leq \Delta^{region}_{t}$

\begin{align}
    M_{t}^{s} (x_t + s) = f_t + g_t^T s + \frac{1}{2} s^T H_t s \quad \text{s.t.} \ \lVert s \rVert \leq \Delta^{region}_{t}
    \label{eq:quad-subproblem}
\end{align}

where $g_t$ is the gradient of the aggregate model at the current iterate $x_t$, $H_t$ is the 
Hessian of the aggregate model at $x_t$, and $\lVert\cdot\rVert$ is the 2-norm of a vector. 
Note that \ref{eq:quad-subproblem} nests the linear model by setting $H_t=0$. 
In the subsequent discussion, we leave the dependence on $x_t$ implicit and drop the subscript $t$ for brevity. %i.e. we write $M^{s} (s)$ instead of $M_{t}^{s} (x_t + s)$.
% Model $M_t$ is based on the Taylor-series expansion of $f$ around $x_t$ 

\paragraph{GQTPAR}

In the unconstrained case, i.e. there are no bounds on the parameter space, GQTPAR solves the trust region
subproblem in \ref{eq:quad-subproblem} almost exactly. It is almost exact since the algorithm does not iterate until a high-accuracy solution is found, but terminates after a few iterations until a fairly loose approximation to the true solution is obtained.\cite{More1983} show that the exact solution $s^*$ of \ref{eq:quad-subproblem} satisfies

% A number of things are immediately apparent about (7.2.1). The solution we are
% seeking lies either interior to the trust region, that is ||s||2 < A, or on the boundary,
% ||s||2 = A. If the solution is interior, the trust-region bound may as well not have
% been there, and therefore SM is the unconstrained minimizer of q(s). But this can only
% happen if q(s) is convex, that is, if the Hessian H is positive semidefinite—we could be
% more precise here, but the details will shortly emerge. At once we see that a minimizer
% of a convex model problem may have a different character from that of a nonconvex
% one. In the nonconvex case, a solution must lie on the boundary of the trust region,
% while in the convex case a solution may or may not do so.


\begin{align}
    ( H + \lambda I ) \ s^* = - g
    \label{eq:exact-solution}
\end{align}

\noindent for some $\lambda \geq 0$, with the matrix $( H + \lambda I )$ being positive semidefinite and 
$\lambda (\Delta^{region}_{t}  - \lVert s^* \rVert) = 0$. The latter is a complementary slackness condition
which states that at least one of the quantities $\lambda$ and $(\Delta^{region}  - \lVert s^* \rVert)$
must be zero at the optimum.
When the solution $s^*$ is interior to the trust region, i.e. $\lVert s^* \rVert < \Delta^{region}$,
then $\lambda^*= 0$ and the algorithm can be terminated immediately.
Otherwise, when $s^*$ lies on the boundary of the trust region, i.e. $\lVert s^* \rVert = \Delta^{region}$,
$\lambda > 0$ and Newton's method is applied to find the value of $\lambda$ such that 
$\lVert s^* \rVert = \Delta^{region}$ is satisfied.
The unique solution is then defined by 

\begin{align}
    s (\lambda) = -(H + \lambda I)^{-1} \ g
    \label{eq:boundary-solution}
\end{align}

\noindent for $\lambda> 0$ sufficiently large so that $H + \lambda I$ is positive definite and 
$ \lVert s (\lambda) \rVert = \Delta^{region}  $. More precisely, Newton's method root finding method is applied to \footnote{Note that Newton's method is not applied to 

\begin{align}
    \phi_0 (\lambda) = \lVert s (\lambda) \rVert - \Delta^{region}
    \label{eq:root-finding}
\end{align}

\noindent directly, because $\phi_0$ may be non-linear around $\lambda^*$. For details, see \cite{More1983} or \cite{Nocedal2006}.}

\begin{align}
    \phi(\lambda)=\frac{1}{\Delta^{region}}-\frac{1}{\|s(\lambda)\|}
\end{align}

\noindent which is almost linear around $\lambda^*$.


\begin{algorithm}
    \caption{GQTPAR algorithm} - The "easy case" \label{algo:gqtpar}
    \KwIn{Initial guess $s_0$, $\lambda^{(0)}$,  $\lambda^{(0)}_L$, $\lambda^{(0)}_U$}
    \For{$\ell$=0,1,2,...}{
            
        Safeguard $\lambda^{(l)}$ to obtain $\lambda^{(\ell)}_S$ \\
        \If{$H + \lambda^{(\ell)} I$ is positive definite}{
            Factor $H + \lambda^{(l)} I = R^T \ R$ \\
            % Solve $R^T \ R \ s_\ell = - g $
            Solve $s_\ell = - (R^T R)^{-1} g $ \\
            % \If{$\lVert s_\ell \rVert < \Delta^{region}$}{
                %     Compute $\tau$ and $\hat{z}$
                % }
                }
                Update $\lambda^{(\ell)}_L$, $\lambda^{(\ell)}_U$, $\lambda^{(\ell)}_S$ \\
                Check convergence criteria \\
                \uIf{$H + \lambda^{(l)} I$ is positive definite and $g \neq 0$ }{
                    Solve $q_\ell = (R^T)^{-1} s_\ell $ \\
                    Set $\lambda^{(\ell+1)}=\lambda^{(\ell)}+\left(\frac{\left\|s_{\ell}\right\|}{\left\|q_{\ell}\right\|}\right)^2\left(\frac{\left\|s_{\ell}\right\|-\Delta^{region}_{t}}{\Delta^{region}_{t}}\right)$
                    }   
                    \uElse{
                        Set $\lambda^{(\ell + 1)} = \lambda^{(\ell)}_S$ 
                        }
                        }
                    \end{algorithm}
                    
\noindent The candidate $s_\ell$ is obtained by factorizing the Hessian matrix via Cholesky factorization and solving the resulting linear system (see lines 5 and 6 of Algorithm \ref{algo:gqtpar}). Note that in line 4, we safeguarding $\lambda^{(\ell)}$. This is necessary to ensure that $H + \lambda^{(\ell)} I$ is positive definite and the Cholesky factorization exists. For details on the safeguarding procedure, see \cite{More1983} or \cite{Nocedal2006}.


\noindent There may situations, however, where $H + \lambda^{(\ell)} I$ is positive definite, but no solution to $ \lVert s (\lambda)  \rVert = \Delta^{region} $ exist. This is what \cite{More1983} call the "hard case", which is not described in Algorithm \ref{algo:gqtpar}. We refer the interested reader to \cite{More1983} and \cite{Conn2000} for details on the hard case. In short, the solution is to find an eigenvector $z$ of $H$ corresponding to the eigenvalue $\lambda_1$. Then

\begin{align}
    \left(H-\lambda_1 I\right)(s+\tau z)=-g
    \label{eq:hard-case}
\end{align}

and $ \lVert s +\tau z \ (\lambda) \rVert  = \Delta^{region}  $ for some $\tau$.


% $\lambda^{*} = ( -\lambda_1, \infty \)$
% $ \lVert s^* \rVert = \delta$
% $ \lVert p \( \lambda \) \rVert  - \Delta = 0 $
% $  \frac{1}{\lVert p \( \lambda \) \rVert}  - \frac{1}{\Delta} = 0 $



\paragraph{Bounded Newton Trust Region (BNTR)}

% BNTR was developed for the Toolkit of Advanced Optimization (TAO) {tao-user-ref} and 
% originally written in C. Tranquilo uses a fast, CPU-accelerated pure-python implementation.
BNTR is a trust region method for bound-constrained optimization of quadratic functions.\footnote{By enforcing box constraints on the parameter space, the spherical trust region constraint \\
$\lVert s \rVert \leq \Delta^{region}_{t}$ in the 2-norm changes to a cubic constraint in an infinity-norm sense.}
These bound, or box, constraints are of the form 
$$\exists i \text{ such that } l_i \leq s_i \leq u_i$$
where $l_i$ and $u_i$ are the lower and upper bounds on the $i$th parameter, respectively. 
Note that not all $s_i$ must, but may, be bounded. For any unbounded parameters, the corresponding inactive lower and upper bounds are set to $-1$ and $1$, the bounds of the cubic trust region, respectively. 

% Internally, BNTR employs two types of projection methods to ensure that the search direction is feasible (cite taoref): (i) gradient projection
% $$
% \mathfrak{P}(g)= \begin{cases}0 & \text { if }\left(s \leq l_i \wedge g_i>0\right) \vee\left(s \geq u_i \wedge g_i<0\right) \\ g_i & \text { otherwise }\end{cases}
% $$
% and (ii) bound projection
% $$
% \mathfrak{B}(s)= \begin{cases}l_i & \text { if } s_i<l_i \\ u_i & \text { if } s_i>u_i \\ s_i & \text { otherwise }\end{cases}
% $$

% BNTR belongs to the class of bounded Newton-Krylov algorithms. 
\noindent BNTR employs a trust region framework combined with an active set approach (\cite{TAO21}) to solve \ref{eq:quad-subproblem}. To ensure that the search direction is feasible, BNTR uses a conjugate gradient method to solve for the minimizer $r$ of the reduced quadratic model (see line 5 of Algorithm \ref{algo:bntr}). \footnote{The available methods are Conjugate Gradient, Steihaug-Toint, and TRSBOX.}
The reduced model is constructed using the reduced model gradient $g^r$ and the reduced model hessian $H^r$ based on the inactive set $\mathcal{I}(x)$. The active and inactive sets are defined as follows (\cite{Bertsekas1982}):
\begin{align*}
\text{lower bounded: } \mathcal{L}(s) & = \{i: s_i \leq l_i + \epsilon \wedge g(s)_i > 0\}, \\
\text{upper bounded: } \mathcal{U}(s) & = \{i: s_i \geq u_i + \epsilon \wedge g(s)_i < 0\}, \\
\text{fixed: } \mathcal{F}(s) & = \{i: l_i = u_i\}, \\
\text{active-set: } \mathcal{A}(s) & = \{\mathcal{L}(s) \cup \mathcal{U}(s) \cup \mathcal{F}(s)\}, \\
\text{inactive-set: } \mathcal{I}(s) & = \{1,2,\ldots,n\} \setminus \mathcal{A}(s).
\end{align*}

\noindent Consider further the bound projection on $s$:

$$
\mathfrak{B}(s)= \begin{cases}l_i & \text { if } s_i<l_i \\ u_i & \text { if } s_i>u_i \\ s_i & \text { otherwise }\end{cases}
$$

\noindent With the reduced conjugate gradient step $r$ in hand, the active parameters in $\mathfrak{B}(s)$ are then set to their respective lower, upper, or fixed bounds, which yields the candidate step $s$ (line 6 of Algorithm \ref{algo:bntr}). The process is repeated until convergence criteria are met.

\begin{algorithm} %[H]
    \caption{BNTR algorithm}\label{algo:bntr}
    \KwIn{Initial guess $s_0$, $\Delta^{sub}_0 > 0$}
    
    Take a finite number of gradient descent steps and update $s_0$, $\Delta^{sub}_0$\\
    \For{$\ell$=0,1,2,...}{

        Create active set of bounds \\
        Construct reduced gradient and reduced model hessian \\
        Solve for the optimum of the reduced model $M^r_\ell$: 
        $r_{\ell}\approx arg\,\min\limits_rM_{\ell}^r(r) \quad \text{s.t.} \ \lVert r \rVert \leq \Delta^{sub}_{\ell}$\\
        Set active dimensions of $r_\ell$ to their respective lower and upper bounds \\
        
        Calculate $\kappa_\ell = \frac{M_{\ell}^r(r_{\ell - 1}) - M_{\ell}^r(r_\ell)}{M_{\ell}^s(r_{\ell - 1}) - M_{\ell}^s(r_\ell)}$, 
        the ratio of predicted over actual reduction 
        
        \uIf{$\kappa_\ell$ is larger than a threshold}{
        Accept step $s_{\ell + 1} = s_\ell + r_\ell$ \\
        Expand trust region radius: $\Delta^{sub}_{\ell+1}=\alpha^{inc}\Delta^{sub}_t$
        }
        
        \uElse{
        Reject step $s_{\ell + 1} = s_\ell$ \\
        Shrink trust region radius $\Delta^{sub}_{\ell+1}=\alpha^{dec}\Delta^{sub}_t$
        }
        Check convergence criteria \\
        }
\end{algorithm}

\begin{itemize}
    \item What do we want to call $x$? Parameters or variables?
    \item Mention that trust region centered at zero vector and radius normalized to 1?
    \item Mention that BNTR was originally developed for the Toolkit of Advanced Optimization (TAO) and
    written in C. Tranquilo uses a fast, CPU-accelerated pure-python implementation.
\end{itemize}
\section{Parallelization}\label{sec:parallelization}

Section introduction:
\begin{itemize}
    \item State our three parallelization categories
        \begin{enumerate}
            \item At start
            \item During sampling
            \item During acceptance step
        \end{enumerate}
    \item List other algorithms that parallelize and how they compare to the three categories
        (DFOLS, Parallel-Neldermead)
    \item Describe the batch size argument and the next multiple function
\end{itemize}

\subsection{Implementation}\label{subsec:parallelization::implementation}

\subsubsection{At Start}

\begin{itemize}
    \item Only in noisy case
    \item Evaluates next multiple of requested evaluations at start
\end{itemize}

\subsubsection{Sampling}

\begin{itemize}
    \item At each point of algorithm (refer to listing)
    \item Instead of sampling requested number of samples, sample next multiple
\end{itemize}


\subsubsection{Acceptance Step}

\begin{itemize}
    \item if batch size is 1 (no parallelization): do classic acceptance based on rhos
    \item else, check if the candidate point is at the tr border
    \item if yes, get number of line search points (at most 3, at least batch size-1) and define the
        grid on the line
    \item calculate number of unallocated function evals based on the number of points on the line
        search and batch size ($batch\_size-1-n_evals_line_search$) \comment[id=MP]{WHY
        $batch\_size-1$ ?}
    \item if the previous number is non-zero, do speculative sampling around the candidate point
        using search radius
    \item add the line search and speculative points, if any, to the history
    \item check if criterion is smaller at any of the new points (spec+line search)
    \item if so, update candidate fval and x
    \item insert algorithm listing at the end
\end{itemize}

\subsection{Benchmarking}\label{subsec:parallelization::benchmarking}

\begin{itemize}
    \item discuss and motivate the cost model
    \item show and discuss parallelization\_ls plot
\end{itemize}
